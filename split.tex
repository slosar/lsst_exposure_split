\documentclass[12pt, a4paper]{article}
\renewcommand{\familydefault}{cmss}
\setlength{\oddsidemargin}{1.5cm}
\setlength{\evensidemargin}{0.5cm}
\setlength{\textwidth}{18.0cm}
\setlength{\parindent}{0cm}
\setlength{\parskip}{0.2cm}
\setlength{\textheight}{24cm}
\setlength{\headheight}{0cm}
\setlength{\headsep}{2cm}
\setlength{\topmargin}{0.0cm}
\addtolength{\topmargin}{-1.1in}
\addtolength{\oddsidemargin}{-1.0in}
\usepackage{setspace}

\def\as#1{[AS: {\it #1}] }


\begin{document}

\title{Proposal to change 2$\times$15s exposures split in LSST}
\author{An\v{z}e Slosar, BNL, your name here}
\maketitle

\section*{Proposal}

We propose to change the current two 15s exposure splitting in the
LSST observing strategy to one where the length of the first exposure
is a randomly chosen time in duration between 2s and 28s and the
length of the second exposure is such that the two add to  30s.

Such strategy will not affect strategy planning outside the 30s window
but will have significant advantages in terms of systematic control
the existing strategy.

\section*{Justification}

\subsection*{Unchanged aspects:}
\begin{enumerate}

\item The overall signal to noise in the 30s is unchanged: there is
  30s worth of integration time, 30s worth of dark current, two servings
  of read-out noise. The same is true for time consumption. In short,
  from SNR perspective, this proposal should be invisible to planning
  outside 30s window.

\item Ability to remove cosmic rays should remain unchanged.
\end{enumerate}
\subsection*{Pros:}

\begin{enumerate}
\item improved dynamic range: objects that would saturate a 15s
  exposure might not saturate a shorter one.

\item ability to study flux dependent systematic effects:
  \begin{itemize}
  \item brighter-fatter effect on PSF
  \item detector non-linearity
  \end{itemize}

\item better transient science:
  \begin{itemize}
  \item sensitivity to coherent oscillation frequencies or noise power at all
    frequencies that are multiples  of $15^{-1}$Hz.
  \item improved sensitivity on events happening at timescales $<15$s 
  \end{itemize}

\item ability to better study separation between readout noise, dark
  current and sky noise. 

\item ability to calibrate the shutter and electronics delays

\item A certain fraction of images will be "lucky images'" (in the
  technical sense of being nearly diffraction limited)

\end{enumerate}

\subsection*{Cons:}

\begin{enumerate}
\item would require better algorithms for differencing two 15s
  exposures, but note that naive differencing would never work due to
  changes in PSF and observing angles over 15s timescales.

\item - would make it easier to study selection effects for transient
  phenomena somewhat more difficult, but note again that these will
  have to be Monte-Carloed anyway.

\item Users working with individual exposures will have one more
  variable aspect to deal with.

\end{enumerate}

\end{document}
