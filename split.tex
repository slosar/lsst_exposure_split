\documentclass[12pt, a4paper]{article}
\usepackage{hyperref}
\renewcommand{\familydefault}{cmss}
\setlength{\oddsidemargin}{1.5cm}
\setlength{\evensidemargin}{0.5cm}
\setlength{\textwidth}{18.0cm}
\setlength{\parindent}{0cm}
\setlength{\parskip}{0.2cm}
\setlength{\textheight}{24cm}
\setlength{\headheight}{0cm}
\setlength{\headsep}{2cm}
\setlength{\topmargin}{0.0cm}
\addtolength{\topmargin}{-1.1in}
\addtolength{\oddsidemargin}{-1.0in}

\usepackage{setspace}

\def\as#1{[AS: {\it #1}] }


\begin{document}
\begin{center}
{\Huge
  Proposal to randomize exposure times for LSST.
}
\end{center}

\begin{center}
David Hogg,
Mike Jarvis,
Andrei Nomerotski,
Paul O'Connor,
An\v{z}e Slosar,
Your Name Here
\end{center}


{\small
Doc repository: \url{https://github.com/slosar/lsst_exposure_split}
}

\section*{Proposal}

We propose to change the current two 15s exposure splitting in the
LSST observing strategy to:

\begin{enumerate}
\item Length of the first exposure is a randomly chosen time in
  duration between 2s and 28s .

\item Length of first exposure is different for every new pair of
  observations and is \emph{not} quantised to the nearest second.

\item Length of the second exposure is such that the two add to 30s.
\end{enumerate}

The 2000 exposures of a typical field will thus uniformly sample
exposure times between 2s and 28s. Such strategy will not affect
strategy planning outside the $2\times 15$s pair window but will have significant
advantages in terms of systematic control over the existing strategy.

\section*{Discussion}

\subsection*{Pros:}

\begin{enumerate}

\item There will be ability to study flux dependent systematic effects:
  \begin{itemize}
  \item brighter-fatter effect on PSF
  \item detector non-linearity
\end{itemize}

\item A convenient new variable to split the data on in order to test for
  uncorrected flux dependent systematics

\item Easier detection of diffraction spikes and CCD bleeeds.

\item The  dynamic range would be improved, raising the bright
  magnitude limit by about one magnitude since objects that would saturate a 15s
  exposure might not saturate a shorter one.

\item The transient science would benefit:
  \begin{itemize}
  \item sensitivity to coherent oscillation frequencies or noise power at all
    frequencies that are multiples  of $15^{-1}$Hz.
  \item improved sensitivity on events happening at timescales $<15$s 
  \end{itemize}

\item It will allow one to calibrate the shutter and electronics delays.

\item A certain fraction of images will be "lucky images'" (in the
  technical sense of being nearly diffraction limited) down to
  CCD diffusion length of 200mas (corresponding to mirror size $\sim$
  6m).

\item It would be easier to understand and separate readout noise, dark
  current and sky noise.

\end{enumerate}

\subsection*{Cons:}

\begin{enumerate}

\item PSF variations will be much more pronounced over very short
  exposures and might require further modeling. 

\item Cosmic rays will be preferentially lie in long exposures,
  thus taking up SNR than in standard scheme, but area lost to cosmic
  rays is expected to be negligible.

\item It would require better algorithms for differencing two 15s
  exposures, but note that naive differencing would never work due to
  changes in PSF and observing angles over 15s timescales.

\item This change would make it more difficult to understand selection
  effects for transient phenomena, but note again that these will have
  to be Monte-Carloed anyway.

\item Users working with individual exposures will have one more
  variable aspect to deal with.

\end{enumerate}

\subsection*{Technical considerations}

\begin{enumerate}
\item The lower limit, set to 2s above is set by the requirement that
  you need to have sufficient integration to be able to find and
  remove cosmic rays and the find sufficient stars in the field to
  characterize the PSF at sufficiently high density across the focal
  plane. The precise lower limit should be evaluated through MC
  simulations.

\item Additional constraint on the lower limit is the minimum exposure
  time set by shutter performance and relative photometric calibration
  requirements. The current SRD specifies minimum exposure to be 5s
  design spec with 1s stretch goal and 10s minimum. But as we note
  above, with many exposures at different times we should be able to
  characterize shutter parameters extremely well.

\item Electronics power dissipation is higher during readout than
  exposure, so there's going to be a small temperature transient at
  the cadence period. If the exposure time is randomized the readouts
  will be seeing different phases of the thermal profile.

\item The overall signal to noise in the 30s is unchanged: there is
  30s worth of integration time, 30s worth of dark current, two
  servings of read-out noise. The same is true for time
  consumption. In short, from SNR perspective, this proposal should be
  invisible to planning outside 30s window. Since the object detection
  will be done on the coadds, it does not matter whether some
  exposures are so short that object is not detected in them -- an
  optimal analysis cannot loose in counting the total number of photons.

\item Ability to remove cosmic rays should remain unchanged with 2s exposures.
\end{enumerate}

\end{document}
